\documentclass[onecolumn, draftclsnofoot,10pt, compsoc]{IEEEtran}
\usepackage{graphicx}
\usepackage{url}
\usepackage{setspace}

\usepackage{geometry}
\geometry{textheight=9.5in, textwidth=7in}

% 1. Fill in these details
%\def \CapstoneTeamName{     The Cleverly Named Team}
%\def \CapstoneTeamNumber{       777}
\def \GroupMemberOne{           Nick Wong}
%\def \GroupMemberTwo{           Thomas Kuhn}
%\def \GroupMemberThree{         Karl Popper}
%\def \CapstoneProjectName{      Building a Robot Using Only Philosophy}
%\def \CapstoneSponsorCompany{   Cheap Robots, Inc}
\def \CapstoneSponsorPerson{        Kevin McGrath}

% 2. Uncomment the appropriate line below so that the document type works
\def \DocType{    %Problem Statement
        %Requirements Document
        %Technology Review
        %Design Document
        %Progress Report
        Writing Assignment 1
        }
      
\newcommand{\NameSigPair}[1]{\par
\makebox[2.75in][r]{#1} \hfil   \makebox[3.25in]{\makebox[2.25in]{\hrulefill} \hfill    \makebox[.75in]{\hrulefill}}
\par\vspace{-12pt} \textit{\tiny\noindent
\makebox[2.75in]{} \hfil    \makebox[3.25in]{\makebox[2.25in][r]{Signature} \hfill  \makebox[.75in][r]{Date}}}}
% 3. If the document is not to be signed, uncomment the RENEWcommand below
%\renewcommand{\NameSigPair}[1]{#1}

%%%%%%%%%%%%%%%%%%%%%%%%%%%%%%%%%%%%%%%
\begin{document}
\begin{titlepage}
    \pagenumbering{gobble}
    \begin{singlespace}
      %\includegraphics[height=4cm]{coe_v_spot1}
        \hfill 
        % 4. If you have a logo, use this includegraphics command to put it on the coversheet.
        %\includegraphics[height=4cm]{CompanyLogo}   
        \par\vspace{.2in}
        \centering
        \scshape{
            \huge CS 444 - Operating Systems II \DocType \par
            {\large\today}\par
            \vspace{.5in}
            %\textbf{\Huge\CapstoneProjectName}\par
            \vfill
            {\large Prepared for}\par
            %\Huge \CapstoneSponsorCompany\par
            \vspace{5pt}
            %{\Large\NameSigPair
            {\CapstoneSponsorPerson\par}
            {\large Prepared by }\par
            \GroupMemberOne\par
            % 5. comment out the line below this one if you do not wish to name your team
            %\CapstoneTeamName\par 
            \vspace{5pt}
            %{\Large
            %    \NameSigPair{\GroupMemberOne}\par
            %    \NameSigPair{\GroupMemberTwo}\par
            %    \NameSigPair{\GroupMemberThree}\par
            %}
            \vspace{20pt}
        }
        \begin{abstract}
        % 6. Fill in your abstract    
           A process can be defined as any one instance of a computer program. Each core in a CPU can only run one process at a time. Threads are a technique which give the impression that multiple processes are being executed at the same time. CPU scheduling is what each CPU uses to organize the execution time of processes. This scheduling allows the CPU to process programs in logical order which results in programs being executed at the right time. These definitions however may slightly be different within the implementations of different operating systems. In this document, the implementation of these terms within the operating systems FreeBSD, and Windows, will be compared and contrasted to the Linux implementation. 

        \end{abstract}     
    \end{singlespace}
\end{titlepage}
\newpage
\pagenumbering{arabic}
\tableofcontents
% 7. uncomment this (if applicable). Consider adding a page break.
%\listoffigures
%\listoftables
\clearpage

% 8. now you write!
\section{Processes}

\subsection{Similarities}

In general, each process in Windows, FreeBSD, and Linux contain program code, signals, and files. According to [11], once a program is loaded into the memory and it becomes a process, it can be divided into four sections, stack, heap, text and data. The stack contains the temporary data, such as method/function parameters, return addresses, and local variables. The heap is composed of dynamically allocated memory and is assigned to a process during its run time. The text includes the current activity represented by the value of the Program Counter register and contents of additional general registers. Finally data contains the global and static variables that are declared in a process.

\subsection{Differences}

Both FreeBSD and Linux call their processes a task, whereas Windows represents its processes as an executive process. According to [9], each task or thread of execution is known as a process. The context of a 4.4BSD process consists of user-level state, the contents of its address space and the run-time environment, and kernel-level state - which includes scheduling parameters, resource controls, and identification information. In Windows, the executive process consists of the Process Environment Block (PEB). Within the PEB, it contains the process heap, thread, storage and executable image to be accessed from the user space. 

One of the main differences between Linux/FreeBSD and Windows is the forming of a new process. In Linux/FreeBSD, according to [1], a new process is created because an existing process makes an exact copy of itself. This child process has the same environment as its parent, only the process ID number is different. This procedure is called forking. After the forking process, the address space of the child process is overwritten with the new process data. This is done through an exec call to the system. Whereas Windows, a CreateProcess function will create a new process, which runs independently of the creating process. However, for simplicity, the relationship is referred to as a parent-child relationship [8].


\subsection{Why these similarities and differences exist}

The very first difference between these three operating systems is the way they create their process. FreeBSD/Linux creates their processes by providing two separate calls in a procedure called forking which are known as tasks. Whereas Windows creates a CreateProcess function which runs independently of the creating process. In a nutshell, FreeBSD/Linux takes an extra step in creating their processes, whereas Windows keeps its process creating function connected. A few similarities to these operating systems is their process address spaces and user mode and kernel mode.

\section{Threads}

\subsection{Similarities}

One similarity that FreeBSD/Linux and Windows share is their implementations of the system and kernel threads. The threads allow for jobs to be executed in kernel mode. According to [4] the executing code has complete and unrestricted access to the underlying hardware. It can execute any CPU instruction and reference any memory address. Kernel mode is generally reserved for the lowest-level, most trusted functions of the OS. Windows, FreeBSD and Linux all support threads that run on the kernel space. FreeBSD and Linux use kernel threads where Windows uses a system thread. All three of these operating systems do not have a task and also do not have address space [5].


\subsection{Differences}
The main difference between Windows/FreeBSD and Linux is that within Linux, there isn't an actual concept of threads. Instead each process can be considered a thread as that is how they are implemented. In addition to this, another main difference between Windows/FreeBSD and Linux is within the application of threads. Another difference is within Windows, where at least one thread has to be present in order for the process it is in to run [2].

\subsection{Why these similarities and differences exist}

Reviewing the similarities and differences between these operating systems result in the idea that they are quite different from each other. The main similarity is that they have threads that allow for jobs to be executed exclusively in kernel mode. However FreeBSD and Linux share thread resources, where in Windows a thread is represented by an executive thread (ETHREAD) block [12]. ETHREADS do not have resources like FreeBSD and Linux. These difference could possibly stem from different ideas of parallel programming. We see that in Linux there is simply no concept of threads, however it uses timing and CPU scheduling (which is discussed later) to implement the idea of parallelism. However when more cores began appearing within CPU's perhaps the concept of parallel programming began to develop more. 

\section{CPU Scheduling}

\subsection{Similarities}

One of the main similarities between Windows, FreeBSD and Linux is their priority based scheduling. According to Microsoft’s website, windows scheduler controls multitasking by determining which of the competing threads receives the next processor time slice. The scheduler determines which thread runs next using scheduling priorities [14]. The default scheduler for FreeBSD is called the ULE. The default ULE assigns a scheduling priority and a CPU by the high-level and low-level scheduler. The high level scheduler determines in which run queue they are placed. In selecting a new thread to run, the low-level scheduler scans the run queues of the CPU needing a new thread from highest to lowest priority and chooses the firs thread on the first non-empty queue [10]. In Linux, the Completely Fair Scheduler (CFS) is a process scheduler which handles CPU resource allocation for executing processes, and aims to maximize overall CPU utilization while also maximizing interactive performance [6].


\subsection{Differences}

Windows uses the Muti-Level Feedback Queue (MLFQ) as their base scheduler [3]. It observes the execution of a job and prioritizes it accordingly. This allows it to deliver excellent overall performance and makes long-running CPU-intensive workloads more efficient. Whereas FreeBSD uses the ULE as its default scheduler, and Linux uses the Completely Fair Scheduler.


\subsection{Why these similarities and differences exist}

All three of these systems have priority based scheduling and has been proven to be one of the best ways in scheduling. With a priority scheduler, the scheduler simply picks the highest priority process to run [13].

\begin{thebibliography}{7}
\bibitem{1} "4.1.5.1. Process creation." Linux - Process creation. N.p., n.d. Web. 28 Apr. 2017.
\bibitem{2} "About Processes and Threads." About Processes and Threads (Windows). N.p., n.d. Web. 29 Apr.
2017. <https://msdn.microsoft.com/en-us/library/windows/desktop/ms681917(v=vs.85).aspx>.
\bibitem{3} Arpaci-Dusseau, Remzi, and Andrea C. Arpaci-Dusseau. Operating systems: three easy pieces. S. l.: Arpaci-Dusseau , 2015. Print.
\bibitem{4} Atwood, Jeff . "Coding Horror." Understanding User and Kernel Mode. N.p., n.d. Web. 28 Apr. 2017. <https://blog.codinghorror.com/understanding-user-and-kernel-mode/>.
\bibitem{5} "Chapter 15. The Process Address Space." Chapter 15. The Process Address Space - Shichao's Notes. N.p., n.d. Web. 29 Apr. 2017. <https://notes.shichao.io/lkd/ch15/>.
\bibitem{6} "Completely Fair Scheduler." Wikipedia. Wikimedia Foundation, 24 Apr. 2017. Web. 27 Apr. 2017.
\bibitem{7} "CPU Scheduling." CPU Scheduling in OS | Operating System Tutorial | Studytonight. N.p., n.d. Web. 27 Apr. 2017. <http://www.studytonight.com/operating-system/cpu-scheduling>.
\bibitem{8} "Creating Processes." Creating Processes (Windows). N.p., n.d. Web. 29 Apr. 2017. <https://msdn.microsoft.com/en-us/library/windows/desktop/ms682512(v=vs.85).aspx>.
\bibitem{9} McKusick, Marshall K., Keith Bostic, Michael J. Karels, and John S. Quarterman. "The Design and Implementation of the 4.4BSD Operating System." 2.4. Process Management. N.p., n.d. Web. 28 Apr. 2017.
\bibitem{10}    McKusick, Marshall Kirk., George V. Neville-Neil, and Robert N. M. Watson. The design and implementation of the FreeBSD operating system. Upper Saddle River: Addison-Wesley/Pearson, 2015. Print.
\bibitem{11}    Tutorialspoint.com. "Operating System - Processes." Www.tutorialspoint.com. Tutorials point, n.d. Web. 28 Apr. 2017.
\bibitem{12}    "Processes, Threads, and Jobs in the Windows Operating System." Processes, Threads, and Jobs in the Windows Operating System | Microsoft Press Store. N.p., n.d. Web. 28 Apr. 2017
\bibitem{13}    Process Scheduling. N.p., n.d. Web. 27 Apr. 2017. <https://www.cs.rutgers.edu/~pxk/416/notes/07-scheduling.html>.
\bibitem{14}    "Scheduling." Scheduling (Windows). N.p., n.d. Web. 28 Apr. 2017. <https://msdn.microsoft.com/en-us/library/windows/desktop/ms685096(v=vs.85).aspx>.

\end{thebibliography}

\end{document}